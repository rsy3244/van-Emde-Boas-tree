\documentclass[dvipdfmx,12pt]{standalone}
\usepackage[subpreambles=true]{standalone}
\usepackage{import}
\usepackage{usermacro}

\begin{document}
\begin{frame}\frametitle{まとめ}
	\begin{itemize}
		\item 平方分割木から, 再帰的な構造であるpvEB構造を考えた.\\
		\item pvEB構造の各操作の時間計算量は, 以下の通りとなった. \\
			\begin{tabular}{|r|l|r|l|}\hline
				操作 & 時間計算量 & 操作 & 時間計算量 \\\hline
				$\Member(V,x)$ & $O(\log \log u)$ & $\Successor(V,x)$ & $O(\log u\log \log u)$ \\
				$\Min(V,x)$    & $O(\log u)$ 	 & $\Predecessor(V,x)$ & $O(\log u\log \log u)$ \\
				$\Max(V,x)$	   & $O(\log u)$	 & $\Insert(V,x)$	& $\Theta(\log u)$ \\
							   &					 & $\Delete(V,x)$	& $O(\sqrt{u}\log \log u)$\\\hline
			\end{tabular}
	\end{itemize}
	\begin{itemize}
		\item pvEB構造に$min$, $max$の変数を持たせたvEB木を考えた. \\
		\item $\Min(V,x)$, $\Max(V,x)$ が定数時間で処理可能となったことから, \\
			他の操作の時間計算量は$O(\log \log u)$ を達成した. \\
	\end{itemize}
\end{frame}


\end{document}
