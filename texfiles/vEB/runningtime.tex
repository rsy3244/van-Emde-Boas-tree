\documentclass[beamer,dvipdfmx,12pt]{standalone}
\usepackage[subpreambles=true]{standalone}
\usepackage{import}
\usepackage{usermacro}
\IfStandalone{
%%%% 和文用 %%%%%
\usepackage{bxdpx-beamer}
\usepackage{pxjahyper}
\usepackage{minijs}%和文用
\usepackage{latexsym}
\usepackage[deluxe,expert]{otf}
\renewcommand{\kanjifamilydefault}{\gtdefault}%和文用
%%%%%%%%

%%%% スライドの見た目 %%%%%
\usetheme{Madrid}
\usefonttheme{professionalfonts}
\setbeamertemplate{frametitle}[default][center]
\setbeamertemplate{navigation symbols}{}
\setbeamercovered{transparent=0}%好みに応じてどうぞ)
\setbeamertemplate{footline}[page number]
\setbeamerfont{footline}{size=\footnotesize,series=\bfseries}
\setbeamercolor{footline}{fg=black,bg=black}
\usepackage{multicol}
\colorlet{mycolor}{orange!80!black}
\usepackage{caption}
\captionsetup[figure]{format=plain, labelformat=empty, labelsep=none}
\renewcommand{\figurename}{}
%%%%%%%%

%%%%% フォント基本設定 %%%%%
\usepackage[T1]{fontenc}%8bit フォント
\usepackage{textcomp}%欧文フォントの追加
\usepackage[utf8]{inputenc}%文字コードをUTF-8
\usepackage{txfonts}%数式・英文ローマン体を txfont にする
\usepackage{bm}%数式太字
%%%%%%%%%%
\usepackage{usermacro}
}{}
\begin{document}
\begin{frame}\frametitle{vEB木 計算量\uniqnum{3}}
    \onslide*<1>{
        \begin{block}{$\Member(V,x)$}
            関数内での再帰呼び出しは$1$回以下なので, 時間計算量は \structure{$O(\log \log  u)$}
        \end{block}
    %}

    %\onslide*<2>{
        \begin{block}{$\Min(V)$}
			各vEBノードは$min$, $max$の値を保持しているので, \\
			時間計算量は\structure{$O(1)$}
        \end{block}
    %}

    %\onslide*<3>{
        \begin{block}{$\Successor(V,x)$}
			関数内では, 再帰呼び出しは常に$1$回以下となる. \\
			また, $\Min(V)$の呼び出しを行うが, これは定数時間で処理可能なので, \\
			$\Successor(V,x)$の時間計算量は\structure{$O(\log \log  u)$}
        \end{block}
    }

    %\onslide*<4>{
	\onslide*<2>{
        \begin{block}{$\Insert(V,x)$}
            $\Insert(V,x)$は$summary$, $cluster$それぞれに\\再帰呼び出しをすることがあるが,\\
			$summary$に対して挿入処理を行うとき, $cluster$への挿入処理は, \\
			その$cluster$の$min$, $max$ を更新しか行わず, \\定数時間で処理可能である. \\
			よって, $\Insert(V,x)$の時間計算量は\structure{$O(\log \log  u)$}

        \end{block}
    }
    %\onslide*<5>{
	\onslide*<3>{
        \begin{block}{$\Delete(V,x)$}
			$\Delete(V,x)$も, $\Insert(V,x)$と同様に, \\$summary$, $cluster$に対し再帰呼び出しをすることがあるが,\\
			$summary$に対して, 削除を行うとき, $cluster$への削除処理は, \\
			その$cluster$の$min$, $max$を$\NIL$に置き換えるだけである. \\
			よって, $\Delete(V,x)$の時間計算量は\structure{$O(\log \log  u)$}
        \end{block}
    }
\end{frame}
\end{document}
