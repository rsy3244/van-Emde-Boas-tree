\documentclass[dvipdfmx,12pt]{standalone}
\usepackage[subpreambles=true]{standalone}
\usepackage{import}

\IfStandalone{
%%%% 和文用 %%%%%
\usepackage{bxdpx-beamer}
\usepackage{pxjahyper}
\usepackage{minijs}%和文用
\usepackage{latexsym}
\usepackage[deluxe,expert]{otf}
\renewcommand{\kanjifamilydefault}{\gtdefault}%和文用
%%%%%%%%

%%%% スライドの見た目 %%%%%
\usetheme{Madrid}
\usefonttheme{professionalfonts}
\setbeamertemplate{frametitle}[default][center]
\setbeamertemplate{navigation symbols}{}
\setbeamercovered{transparent=0}%好みに応じてどうぞ)
\setbeamertemplate{footline}[page number]
\setbeamerfont{footline}{size=\footnotesize,series=\bfseries}
\setbeamercolor{footline}{fg=black,bg=black}
\usepackage{multicol}
\colorlet{mycolor}{orange!80!black}
\usepackage{caption}
\captionsetup[figure]{format=plain, labelformat=empty, labelsep=none}
\renewcommand{\figurename}{}
%%%%%%%%

%%%% 定義環境 %%%%%
\usepackage{amsmath,amssymb}
\usepackage{amsthm}
\theoremstyle{definition}
\newtheorem{theorem}{定理}
\newtheorem{definition}{定義}
\newtheorem{proposition}{命題}
\newtheorem{lemma}{補題}
\newtheorem{corollary}{系}
\newtheorem{conjecture}{予想}
\newtheorem*{remark}{Remark}
\renewcommand{\proofname}{}
%%%%%%%%%

%%%%% フォント基本設定 %%%%%
\usepackage[T1]{fontenc}%8bit フォント
\usepackage{textcomp}%欧文フォントの追加
\usepackage[utf8]{inputenc}%文字コードをUTF-8
\usepackage{txfonts}%数式・英文ローマン体を txfont にする
\usepackage{bm}%数式太字
%%%%%%%%%%

\usepackage{usermacro}
}{}

\begin{document}
\begin{frame}\frametitle{van Emde Boas tree}
\onslide*<1>{
	pvEB構造に\structure{最小値}, \structure{最大値}を加えたvan Emde Boas treeを考える.
	\begin{block}{van Emde Boas tree}
	    van Emde Boas tree ( vEB木 ) は, \\
	    van Emde Boas node ( vEBノード) で構成されるデータ構造\\
		\begin{itemize}
			\item 各vEBノードはpvEB構造と同様の変数に加え, \\
			そのノードが持つ要素の最小値, 最大値を保持
			\item $u = 2$の場合は, ポインタを持たず, 最小値, 最大値のみを保持\\
			\item 最小値の要素は, 子のノードに持たせない\\
		\end{itemize}
	\end{block}
}
\onslide*<2>{
	\begin{figure}\resizebox{\textwidth}{!}{
			\subimport{../../resources/vEB/}{node}
		}
		\caption{van Emde Boas tree \\$V = \{2,3,5,8,11\}$}
	\end{figure}
}
\end{frame}
\end{document}
