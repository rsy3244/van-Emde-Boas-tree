\documentclass[beamer,dvipdfmx,12pt]{standalone}
\usepackage[subpreambles=true]{standalone}
\usepackage{import}
\usepackage{usermacro}
\IfStandalone{
%%%% 和文用 %%%%%
\usepackage{bxdpx-beamer}
\usepackage{pxjahyper}
\usepackage{minijs}%和文用
\usepackage{latexsym}
\usepackage[deluxe,expert]{otf}
\renewcommand{\kanjifamilydefault}{\gtdefault}%和文用
%%%%%%%%

%%%% スライドの見た目 %%%%%
\usetheme{Madrid}
\usefonttheme{professionalfonts}
\setbeamertemplate{frametitle}[default][center]
\setbeamertemplate{navigation symbols}{}
\setbeamercovered{transparent=0}%好みに応じてどうぞ)
\setbeamertemplate{footline}[page number]
\setbeamerfont{footline}{size=\footnotesize,series=\bfseries}
\setbeamercolor{footline}{fg=black,bg=black}
\usepackage{multicol}
\colorlet{mycolor}{orange!80!black}
\usepackage{caption}
\captionsetup[figure]{format=plain, labelformat=empty, labelsep=none}
\renewcommand{\figurename}{}
%%%%%%%%

%%%%% フォント基本設定 %%%%%
\usepackage[T1]{fontenc}%8bit フォント
\usepackage{textcomp}%欧文フォントの追加
\usepackage[utf8]{inputenc}%文字コードをUTF-8
\usepackage{txfonts}%数式・英文ローマン体を txfont にする
\usepackage{bm}%数式太字
%%%%%%%%%%
\usepackage{usermacro}
}{}
\begin{document}
\begin{frame}\frametitle{vEB木 操作: $\Insert(V,x)$}
    $min$が$\NIL$の場合, $min$と$max$を$x$に更新して終了する.\\
    $x$が$min$より小さければ$x$と$min$を交換し,\\ 今後$min$であった値について更新する.\\
    対応する$cluster$の最小値が$\NIL$の場合, $\Insert(summary, \Func{high}(x))$を行う.\\
    $\Insert(cluser[\Func{high}(x)], \Func{low}(x))$を実行し終了する.
    

	\begin{figure}\resizebox{\textwidth}{!}{
			\subimport{../../resources/vEB/}{insert}
		}
		\caption{$\Insert(V,4), V = \{2,3,5,8,11\}$}
	\end{figure}
	\begin{tikzpicture}[overlay]
		%\draw (0,0) grid (30,30);
		%\node at (5,0) {5};
		%\node at (0,5) {5};
		%\node at (0,0) {\uniqnum{9}};
	    \onslide*<1>{\pic at (6,5) {myCallout={$\Insert(root, 4)$ dir 3,0}};}
		\onslide*<2>{\pic at (6,5) {myCallout={$\Insert(cluster[1], 0)$ dir 3,0}};}
		\onslide*<4>{\pic at (7,4.3) {myCallout={$\Min(cluster[0])$ dir 1,1}};}
		\onslide*<6>{\pic at (7,4.3) {myCallout={$\Insert(summary, 0)$ dir 1,1}};}
		\onslide*<8>{\pic at (7,4.3) {myCallout={$\Insert(cluster[0], 0)$ dir 1,1}};}
	\end{tikzpicture}
\end{frame}
\end{document}
