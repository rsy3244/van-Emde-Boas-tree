\documentclass[beamer,dvipdfmx,12pt]{standalone}
\usepackage[subpreambles=true]{standalone}
\usepackage{import}
\usepackage{usermacro}
\IfStandalone{
%%%% 和文用 %%%%%
\usepackage{bxdpx-beamer}
\usepackage{pxjahyper}
\usepackage{minijs}%和文用
\usepackage{latexsym}
\usepackage[deluxe,expert]{otf}
\renewcommand{\kanjifamilydefault}{\gtdefault}%和文用
%%%%%%%%

%%%% スライドの見た目 %%%%%
\usetheme{Madrid}
\usefonttheme{professionalfonts}
\setbeamertemplate{frametitle}[default][center]
\setbeamertemplate{navigation symbols}{}
\setbeamercovered{transparent=0}%好みに応じてどうぞ)
\setbeamertemplate{footline}[page number]
\setbeamerfont{footline}{size=\footnotesize,series=\bfseries}
\setbeamercolor{footline}{fg=black,bg=black}
\usepackage{multicol}
\colorlet{mycolor}{orange!80!black}
\usepackage{caption}
\captionsetup[figure]{format=plain, labelformat=empty, labelsep=none}
\renewcommand{\figurename}{}
%%%%%%%%

%%%%% フォント基本設定 %%%%%
\usepackage[T1]{fontenc}%8bit フォント
\usepackage{textcomp}%欧文フォントの追加
\usepackage[utf8]{inputenc}%文字コードをUTF-8
\usepackage{txfonts}%数式・英文ローマン体を txfont にする
\usepackage{bm}%数式太字
%%%%%%%%%%
\usepackage{usermacro}
}{}
\begin{document}
\begin{frame}\frametitle{pvEB構造 操作: $\Successor(V,x)$}
	$\Successor(cluster[\Func{high}(x)],\Func{low}(x))$を呼び出し, \\
    返り値が$\NIL$なら, $\Successor(summary, high(x))$で, \\
    次に要素を持つ$cluster$を求め, その最小値を返す.\\
    そうでないなら, 返り値に対応した値を返す.

	\begin{figure}\resizebox{\textwidth}{!}{
			\subimport{../../resources/pvEB/}{successor}
		}
		\caption{$\Successor(V,5), V = \{2,3,5,8,11\}$}
	\end{figure}
	\begin{tikzpicture}[overlay]
	    \onslide*<1>{
    		%\node[ellipse callout, thick, draw, fill=white, callout absolute pointer={(6,4.5)}] at (7,3.5) {$\Member(V,5)$};
    		%\pic ["$\Member(V,5)$"] at (6,4.5) {myCallout};
		}
	\end{tikzpicture}
\end{frame}
\end{document}
