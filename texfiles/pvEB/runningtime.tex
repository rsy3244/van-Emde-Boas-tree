\documentclass[beamer,dvipdfmx,12pt]{standalone}
\usepackage[subpreambles=true]{standalone}
\usepackage{import}
\usepackage{usermacro}
\IfStandalone{
%%%% 和文用 %%%%%
\usepackage{bxdpx-beamer}
\usepackage{pxjahyper}
\usepackage{minijs}%和文用
\usepackage{latexsym}
\usepackage[deluxe,expert]{otf}
\renewcommand{\kanjifamilydefault}{\gtdefault}%和文用
%%%%%%%%

%%%% スライドの見た目 %%%%%
\usetheme{Madrid}
\usefonttheme{professionalfonts}
\setbeamertemplate{frametitle}[default][center]
\setbeamertemplate{navigation symbols}{}
\setbeamercovered{transparent=0}%好みに応じてどうぞ)
\setbeamertemplate{footline}[page number]
\setbeamerfont{footline}{size=\footnotesize,series=\bfseries}
\setbeamercolor{footline}{fg=black,bg=black}
\usepackage{multicol}
\colorlet{mycolor}{orange!80!black}
\usepackage{caption}
\captionsetup[figure]{format=plain, labelformat=empty, labelsep=none}
\renewcommand{\figurename}{}
%%%%%%%%

%%%%% フォント基本設定 %%%%%
\usepackage[T1]{fontenc}%8bit フォント
\usepackage{textcomp}%欧文フォントの追加
\usepackage[utf8]{inputenc}%文字コードをUTF-8
\usepackage{txfonts}%数式・英文ローマン体を txfont にする
\usepackage{bm}%数式太字
%%%%%%%%%%
\usepackage{usermacro}
}{}
\begin{document}
\begin{frame}\frametitle{pvEB構造 計算量\uniqnum{5}}
    \onslide*<1>{
        \begin{block}{空間計算量}
            pvEB構造は自身の$u$について, $\sqrt{u}+1$個のポインタを持つ.\\
            よって, 全体の空間計算量を$\Func{S}(u)$とすると, \\
            \begin{equation*}
            \Func{S}(u) = (\sqrt{u}+1)\Func{S}(\sqrt{u}) + \Theta(\sqrt{u})
            \end{equation*}
            となる. よって, 以下が成り立つ.\\
            \begin{equation*}
				\Func{S}(u) \leq \left( \prod_{i=1}^{\Func{loglog}u} (u^{2^{-i}}+1)\right)\Func{S}(2) + \sum_{i=1}^{\Func{loglog}u}\Theta(u^{2^{-i}})(u^{2^{-i}}+1)
            \end{equation*}
            
        \end{block}
    }

    \onslide*<2>{
        \begin{block}{$\Member(V,x)$}
            $\Member(V,x)$は, 根に相当するpvEB構造から葉まで探索するので, 時間計算量は \structure{$O(\Func{loglog}u)$}
        \end{block}
    }

    \onslide*<3>{
        \begin{block}{$\Min(V)$}
            $\Min(V)$は, 時間計算量を$\Func{T}(u)$とすると, 以下のように導出できる. \\
            \begin{eqnarray*}
                \Func{T}(u) &=& 2\Func{T}(\sqrt{u}) + O(1) \\
                \Func{S}(m) &:=& \Func{T}(2^{m}) \text{とすると,}\\
                \Func{S}(m) &=& 2\Func{S}\left(\frac{m}{2} \right) + O(1)\\
                \Func{S}(m) &=& \Theta(m)\\
                \Func{T}(u) &=& \Func{T}(2^{m}) = \Func{S}(m) = \Theta(m) = \alert{\Theta(\log u)}\\
                %master method
            \end{eqnarray*}
        \end{block}
    }

    \onslide*<4>{
        \begin{block}{$\Successor(V,x)$}
            $\Successor(V,x)$は, $1$回の処理で, $\Successor(summary,high(x))$,\\
            $\Successor(cluster[high(x)], low(x))$を呼び出す. \\
            また, $\Min(V')$も呼び出すので時間計算量$\Func{T}(u)$は以下の式となる.
            $\Min(V)$と同様に導出する. 
            \begin{eqnarray*}
                \Func{T}(u) &=& 2\Func{T}(\sqrt{u}) + \Theta(\Func{log}u)\\
                \Func{S}(m) &=& 2\Func{S}\left( \frac{m}{2} \right) + \Theta(m)\\
                \Func{S}(m) &=& O(m\log m) \\
                \Func{T}(u) &=& \Func{S}(m) = O(m\log m) = \alert{O(\log u\log \log u)}\\
            \end{eqnarray*}
        \end{block}
    }

    \onslide*<5>{
        \begin{block}{$\Insert(V,x)$}
            $\Insert(V,x)$は$summary$, $cluster$それぞれに再帰的に処理を行うので, \\
            $\Min(V)$と同様に時間計算量は\alert{$\Theta(\Func{log}u)$}となる.
        \end{block}
    }

    \onslide*<5>{
        \begin{block}{$\Delete(V,x)$}
            $\Delete(V,x)$は, $cluster$で更新後, \\
            $cluster$に要素がないか確認する必要がある. \\
            $cluster$は$\sqrt{u}$個の要素を持ち, \\
            それぞれに$\Member(cluster[\Func{high}(x)],i)$で確認するので,\\
            時間計算量は\alert{$O(\sqrt{u}\Func{loglog}u)$}となる.
        \end{block}
    }
\end{frame}
\end{document}
