\documentclass[dvipdfmx,12pt]{standalone}
\usepackage[subpreambles=true]{standalone}
\usepackage{import}

\begin{document}
\begin{frame}\frametitle{proto van Emde Boas structures}
\onslide*<1>{
	動的集合$v$を保持するproto van Emde Boas structuresを考える.
	\begin{block}{proto van Emde Boas structures}
		proto van Emde Boas structures (pvEB構造)とは, \\
		$u$の値と$summary$, $cluster$に対応する\\ 
		pvEB構造へのポインタを持つデータ構造
		\begin{itemize}
			\item ポインタの先のpvEB構造は$\sqrt{u}$個の要素を保持\\
			\item $u = 2$の場合は, ポインタを持たず, $\{0, 1\}$の要素を保持\\
			\item pvEB構造間で対応する要素の値が異なるので, \\
				以下の式を用いてやり取りを行う.\\
				\begin{description}[$index(x,y)$]
					\item[$index(x,y)$] こんなかんじ\\
				\end{description}
		\end{itemize}
	\end{block}
}
\onslide*<2>{
	平方分割木の$summary$, $cluster$の配列を \\
	$u=4$のpvEB構造に置き換える

	\begin{figure}\resizebox{\textwidth}{!}{
			\subimport{../../resources/pvEB/}{node}
		}
		\caption{proto van Emde Boas structures \\$V = \{2,3,5,8,11\}$}
	\end{figure}
}
\end{frame}
\end{document}
