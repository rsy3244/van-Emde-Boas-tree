\documentclass[beamer,dvipdfmx,12pt]{standalone}
\usepackage[subpreambles=true]{standalone}
\usepackage{import}
\usepackage{usermacro}
\IfStandalone{
%%%% 和文用 %%%%%
\usepackage{bxdpx-beamer}
\usepackage{pxjahyper}
\usepackage{minijs}%和文用
\usepackage{latexsym}
\usepackage[deluxe,expert]{otf}
\renewcommand{\kanjifamilydefault}{\gtdefault}%和文用
%%%%%%%%

%%%% スライドの見た目 %%%%%
\usetheme{Madrid}
\usefonttheme{professionalfonts}
\setbeamertemplate{frametitle}[default][center]
\setbeamertemplate{navigation symbols}{}
\setbeamercovered{transparent=0}%好みに応じてどうぞ)
\setbeamertemplate{footline}[page number]
\setbeamerfont{footline}{size=\footnotesize,series=\bfseries}
\setbeamercolor{footline}{fg=black,bg=black}
\usepackage{multicol}
\colorlet{mycolor}{orange!80!black}
\usepackage{caption}
\captionsetup[figure]{format=plain, labelformat=empty, labelsep=none}
\renewcommand{\figurename}{}
%%%%%%%%

%%%%% フォント基本設定 %%%%%
\usepackage[T1]{fontenc}%8bit フォント
\usepackage{textcomp}%欧文フォントの追加
\usepackage[utf8]{inputenc}%文字コードをUTF-8
\usepackage{txfonts}%数式・英文ローマン体を txfont にする
\usepackage{bm}%数式太字
%%%%%%%%%%
\usepackage{usermacro}
}{}
\begin{document}
\begin{frame}\frametitle{pvEB構造 欠点}
    \begin{itemize}
        \item $\Min(V)$は, $summary$, $cluster$両方に対し再帰的に処理\\
        \onslide+<2->{\hspace{2em}$\to$\structure{再帰呼び出しは$1$回まで}\\}
        \item $\Successor(V,x)$は, 上に加え$\Min(V,x)$を処理\\
        \onslide+<2->{\hspace{2em}$\to$\structure{再帰呼び出しは$1$回まで}\\}
        \item $\Insert(V,x)$は, $summary$, $cluster$両方を再帰的に更新\\
        \onslide+<3->{\hspace{2em}$\to$\structure{更新処理を最低限に}\\}
        \item $\Delete(V,x)$は, $cluster$の線形処理がボトルネック\\
        \onslide+<4->{\hspace{2em}$\to$\structure{要素がないかを定数時間で判定}\\}
    \end{itemize}
\end{frame}
\end{document}
