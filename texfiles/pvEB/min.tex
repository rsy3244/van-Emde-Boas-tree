\documentclass[beamer,dvipdfmx,12pt]{standalone}
\usepackage[subpreambles=true]{standalone}
\usepackage{import}
\usepackage{usermacro}
\IfStandalone{
%%%% 和文用 %%%%%
\usepackage{bxdpx-beamer}
\usepackage{pxjahyper}
\usepackage{minijs}%和文用
\usepackage{latexsym}
\usepackage[deluxe,expert]{otf}
\renewcommand{\kanjifamilydefault}{\gtdefault}%和文用
%%%%%%%%

%%%% スライドの見た目 %%%%%
\usetheme{Madrid}
\usefonttheme{professionalfonts}
\setbeamertemplate{frametitle}[default][center]
\setbeamertemplate{navigation symbols}{}
\setbeamercovered{transparent=0}%好みに応じてどうぞ)
\setbeamertemplate{footline}[page number]
\setbeamerfont{footline}{size=\footnotesize,series=\bfseries}
\setbeamercolor{footline}{fg=black,bg=black}
\usepackage{multicol}
\colorlet{mycolor}{orange!80!black}
\usepackage{caption}
\captionsetup[figure]{format=plain, labelformat=empty, labelsep=none}
\renewcommand{\figurename}{}
%%%%%%%%

%%%%% フォント基本設定 %%%%%
\usepackage[T1]{fontenc}%8bit フォント
\usepackage{textcomp}%欧文フォントの追加
\usepackage[utf8]{inputenc}%文字コードをUTF-8
\usepackage{txfonts}%数式・英文ローマン体を txfont にする
\usepackage{bm}%数式太字
%%%%%%%%%%
\usepackage{usermacro}
}{}
\begin{document}
\begin{frame}\frametitle{pvEB構造 操作: $\Min(V)$}
	$summary$に対して$\Min(V.summary)$を呼び出し, \\
	返り値を$mincluster$とすると,\\
	$\Min(V.cluster[mincluster])$を再帰的に呼び出す.\\
	再帰関数の返り値$ret$はそのpvEB構造に対応した値なので,\\
	返す値は, $\Func{index}(mincluster, ret)$となる.

	\begin{figure}\resizebox{\textwidth}{!}{
			\subimport{../../resources/pvEB/}{min}
		}
		\caption{$\Min(V,5), V = \{2,3,5,8,11\}$}
	\end{figure}
	\begin{tikzpicture}[overlay]
		\draw (0,0) grid (30,30);
	    \onslide*<1>{\pic at (6,4.8) {myCallout={$\Min(root)$ dir 2,0}};}
	    \onslide*<2>{\pic at (6,4.8) {myCallout={$\Min(summary)$ dir 3.5,0}};}
	    \onslide*<3>{\pic at (2.9,4) {myCallout={$\Min(summary)$ dir 2,1}};}
		\onslide*<5>{\pic at (2.9,4) {myCallout={$\Min(cluster[0])$ dir 2,1}};}
		\onslide*<7>{\pic at (6,4.8) {myCallout={$\Min(cluster[0])$ dir 2,0}};}
	\end{tikzpicture}
\end{frame}
\end{document}
