\documentclass[dvipdfmx,12pt]{standalone}
\usepackage[subpreambles=true]{standalone}
\usepackage{import}
\usepackage{usermacro}

\begin{document}
\begin{frame}\frametitle{前回の復習}
	\begin{itemize}
		\item vEB木を定義した.
		\item 二分木構造で各種操作の時間計算量は下表となった.
		\item 平方分割木によって二分木構造の木の高さを小さくしたが, \\
			一部操作の最悪時間計算量が$O(\sqrt{u})$と悪化\\
			\begin{itemize}
				\item $summary$, $cluster$の線形探索がボトルネック
			\end{itemize}
	\end{itemize}
	\begin{table}
		\begin{tabular}{|c|c|c|}\hline
			& 二分木 & 平方分割木 \\\hline
			{\scriptsize $\Min(V)$ } 			& \structure{ $\Theta(\log u)$} 		& \alert{$O(\sqrt{u })$} \\
			{\scriptsize $\Max(V)$ } 			& \structure{ $\Theta(\log u)$} 		& \alert{$O(\sqrt{u} )$} \\
			{\scriptsize $\Member(V,x)$ } 		& $O(1)$ 					& $O(1)$ \\
			{\scriptsize $\Successor(V,x)$ } 	& \structure{ $O(\log u)$} 	& \alert{$O(\sqrt{u})$} \\
			{\scriptsize $\Predecessor(V,x)$ } 	& \structure{ $O(\log u)$}	& \alert{$O(\sqrt{u})$} \\
			{\scriptsize $\Insert(V,x)$ } 		& \alert{$O(\log u)$}		& \structure{$O(1)$} \\
			{\scriptsize $\Delete(V,x)$ } 		& \structure{ $O(\log u)$}	& \alert{ $O(\sqrt{u})$} \\\hline
		\end{tabular}
	\end{table}
\end{frame}

\begin{frame}\frametitle{平方分割木 構造}
\begin{block}{平方分割木}
	平方分割木とは, 葉を$\sqrt{u}$ 個持ち, \structure{高さが$2$}の, \\以下のような特徴を持つデータ構造である.\\
	\begin{itemize}
	\item 要素の全体集合の大きさ$u$を$u = \structure{2^{2k}} (k\text{は非負整数})$ とする.\\
	\item 配列を$\sqrt{u}$個に分割し, それぞれを\structure{$cluster$}とする.\\
	\item $cluster$は大きさが$\sqrt{u}$の配列であり, 対応する葉の値を持つ.\\
	\item 根は大きさ$\sqrt{u}$の配列を持ち (これを\structure{$summary$}とする) ,\\ それぞれの要素は各$cluster$の要素の論理和を保持する.
	\end{itemize}
\end{block}
\subimport{../resources/sqrt-tree/}{sqrt-tree}
\end{frame}


\end{document}
