\documentclass[dvipdfmx,12pt]{standalone}
\usepackage[subpreambles=true]{standalone}

%%%% Tikz環境 %%%%
\usepackage{tikz}
\usetikzlibrary{calc,decorations.pathreplacing,quotes,positioning,shapes,fit,arrows,backgrounds,tikzmark}
%%%%%%%%
%\ifCOSTUMMACRO
%\else
%	\ProvidesPackage{usermacro}
%%%% 関数や定数マクロ %%%%
\newcommand*{\COSTUMMACRO}{}
\newcommand{\Func}[1]{\ensuremath\mathrm{#1}}
\newcommand{\Min}{\Func{\text{M\textsc{in}}}}
\newcommand{\Max}{\Func{\text{M\textsc{ax}}}}
\newcommand{\Member}{\Func{\text{M\textsc{ember}}}}
\newcommand{\Successor}{\Func{\text{S\textsc{uccessor}}}}
\newcommand{\Predecessor}{\Func{\text{P\textsc{redecessor}}}}
\newcommand{\Insert}{\Func{\text{I\textsc{nsert}}}}
\newcommand{\Delete}{\Func{\text{D\textsc{elete}}}}
\newcommand{\TRUE}{\text{\textsc{true}}}
\newcommand{\FALSE}{\text{\textsc{false}}}
\newcommand{\NIL}{\text{\textsc{nil}}}
\newcommand{\uniqnum}[1]{
	(\foreach \i in {1,2,...,#1} {\onslide*<\i>{\i}}/#1)
}
\newcounter{Cntr}
\newcommand{\putarray}[2][0]{
	\setcounter{Cntr}{#1}
	\foreach \i in {#2} {
		\node (a\the\value{Cntr}) at ({\the\value{Cntr}-#1+.5},.5) {$\i$};
		\node (idx\the\value{Cntr}) at ({\the\value{Cntr}-#1+.5},-.25) {{\tiny$\the\value{Cntr}$}};
		\addtocounter{Cntr}{1}
	%	\node (idx\i) [above =of a\i] {{\i-1}};
	}
}
%%%%%%%%
\newcounter{xbase}

%\fi
\usepackage{usermacro}

\begin{document}
\begin{tikzpicture}[scale=0.5,
	elem/.style={rectangle, fill=blue!10},
	a0/.style={elem},
	a1/.style={elem},
	a2/.style={elem},
	a3/.style={elem},
	a8/.style={elem},
	a9/.style={elem},
	a10/.style={elem},
	a11/.style={elem},
	]
	%\draw (0,0) grid (16,1);
	\putarray{0,0,1,1,0,1,0,0,1,0,0,1,0,0,0,0}
	\setcounter{Cntr}{0}
	\foreach \i in {1,1,1,0} {
		\node (r\the\value{Cntr}) at ({\the\value{Cntr}+6.5},3) {$\i$};
		\node [above =-.1 of r\the\value{Cntr}] (rdx\the\value{Cntr}) {{\tiny$\the\value{Cntr}$}};
		\addtocounter{Cntr}{1}%
	}
	\draw [thick] (a0.north west) rectangle (a15.south east);
	\begin{scope}[on background layer]
		\draw [fill=blue!5] (a0.north west) rectangle (a3.south east);
		\draw [fill=blue!5] (a8.north west) rectangle (a11.south east);
	\end{scope}
	\foreach \i in {0,1,...,14} {
		\draw (a\i.north east) to (a\i.south east);
	}
	\draw (r0.north west) rectangle (r3.south east);
	\foreach \i in {0,1,2} {
		\draw (r\i.north east) to (r\i.south east);
	}
	\draw (r0.south) to (a1.north east);
	\draw (r1.south) to (a5.north east);
	\draw (r2.south) to (a9.north east);
	\draw (r3.south) to (a13.north east);
	\draw [thick](a4.north west) to (a4.south west);
	\draw [thick](a8.north west) to (a8.south west);
	\draw [thick](a12.north west) to (a12.south west);

	\draw[thick, decorate, decoration={brace,mirror,amplitude=.3cm}] (idx0.south west) -- (idx3.south east) node[midway,yshift=-.2cm](vertex){};
	
	\node(cr0) at (6,-3) {$0$};
	\node(cr1) at (7,-3) {$1$};
	\draw [thick] (cr0.north west) rectangle (cr1.south east);
	\setcounter{Cntr}{0}
	\foreach \i in {0,0,1,1} {
		\node (ccr\the\value{Cntr}) at ({\the\value{Cntr}+5},-6) {$\i$};
		\addtocounter{Cntr}{1}%
	}
	\begin{scope}[on background layer]
		\draw [fill=blue!5] (ccr0.north west) rectangle (ccr1.south east);
	\end{scope}
	\draw(cr0.north east) to (cr0.south east);
	\draw [thick] (ccr0.north west) rectangle (ccr3.south east);
	\draw(ccr0.north east) to (ccr0.south east);
	\draw[thick] (ccr1.north east) to (ccr1.south east);
	\draw(ccr2.north east) to (ccr2.south east);
	\draw(cr0.south) to (ccr0.north east);
	\draw(cr1.south) to (ccr2.north east);
	\draw(vertex.south) to (ccr0.south west);
	\draw(vertex.south) to (cr1.north east);
	\node[rectangle callout, draw, callout absolute pointer={($(ccr3.north east) + (0,1)$)}] at ($(ccr3) + (10,0)$) {\begin{tabular}{l}それぞれの$cluster$を管理する\\小さい平方分割木を考える\end{tabular}};
	
\end{tikzpicture}
\end{document}